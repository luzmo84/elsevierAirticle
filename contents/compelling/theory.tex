\subsection*{}
Una vez demostrada la relevancia, beneficios y ventajas de la utilizacion de los datos por parte del usuario, es necesario un trabajo minucioso para
transmitir estos conceptos al usuario de forma clara, sencilla y directa y asi lograr un vinculo que haga crecer el deseo del usuario de profundizar en la materia y
mantenerse informado. 
Solo si el usuario entiende la relevancia de los datos y la influencia que estos tienen en su vida diaria, sera posible un compromiso por parte de este.

Para conseguir este objetivo, es importante reflejar la informacion de una manera atractiva tanto desde un punto de vista conceptual como funcional, 
es decir, dejando claro la implicacion de los datos y reflejandolos con una correcta presentacion, que guie al usuario de una manera logica y estructurada.

Desde un punto de vista conceptual, la informacion debera cubrir una necesidad del usuario, aunque el usuario no sepa a priori que tiene esta necesidad.
Es decir, aqui entra en juego la representacion de la informacion de una manera que despierte la necesidad de monitorizacion de los datos por parte
del usuario.

[....].

--mision: atraer usuarios, crear un valor agregado, satisfaccion del usuario
--crear una relacion con el producto
--diferenciacion del producto 
\begin{itemize}

\item usuario no sabe que tiene la necesidad
\item necesidad para el usuario
\item fundamental para la vida diaria
\item mejora para sus tareas diarias
\item personalizacion
\item Ligar la informacion a la resolucion de problemas comunes
\item \end{itemize}

Desde un punto de vista funcional se ha de tener en cuante, que para lograr el mayor alcance de usuarios, de distintos tipos, se han
de repetar las siguientes premisas,  la informacion debe ser presentada en un formato facil de entender por
el usuario, de una forma logica, que respete un order, es decir, de menos detalles a mas, que le de la posibilidad al usuario de indagar por si mismo, como
si la redaccion de una novela se tratara.
La representacion debe estar fuera de toda ambiguidad, no mostrar informacion que resulte enganosa o que haya que dedicar demasiado tiempo fijandonos
en los detalles para poder interpretarla correctamente.
Ademas se debera buscar la manera de presentar la informacion de una manera armonica y agradable para el usuario.
Al represnentar la informacion sera incluido lo basicamente necesario, huyendo siempre de graficas recargadas que solo aporten ruido a la represetacion.


\begin{itemize}
\item facil de entender
\item Analisis del publico al que va dirijido
\item agradable
\item no enganar
\item nivel alto de detalle sin datos personales
\item actual
\item Herramientas necesarias para que el usuario entienda la informacion y de porque
\item formato adecuado para un amplio espectro de usuarios
\item Crear familiaridad
\item Flexibilidad de seleccion de subdata
\item No decoracion que ensucie o distraiga de la informacion principal
\item Consecuente, que siga una secuencia logica
\item Estudio de las representaciones mas adecuadas para cada tipo de datos
  \end{itemize}

  Empatizar con el usuario
  --Proteccion de su intimidad
  --Minimizacion de captura de datosSiempre pedir permiso
  Asegurar seguridad, integridad y confidencialidad
  Transparencia, opcion a eliminar los datos
  Explicar finalidad de los datos\\

  