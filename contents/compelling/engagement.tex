\subsection{Engagement}
It is a truism that you can lead a horse to water, but you can't make it drink. \\

So it is with our users.
No matter how easy to access, how important, and how personalised the information, if we cannot engage and maintain the user's interest,
they will likely ignore the information we provide, possibly to their significant detriment.
We must think like salespeople, like bestselling authors, like ad-agency strategists.
This is often a difficult shift in mindset for scientists and engineers to make.
Our attitude to sales and marketing ranges from begruding acceptance to violent hostility.

It is important to recognise the difference between ourselves and our motivations, and our the motivations of our users. Whilst we engage with the data in a scientific
manner, the users do not. They have no inherent interest. In fact, it is quite possible that our own interest is less in the information per-se, and more in the process
by which we come to it, the analysis techniques, the grant and publication opportunities. We are all considerably less rational than we think, and more motivated by
emotional impulses. We may wish the world to be a rational place driven by the cool dictates of reason, but wishing does not make it so.

It is well known in advertising, PR, and politics, that emotion easily trumps reason - these fields have turned the art of emotional engagement into a scientific discipline.
As have companies like Facebook, Apple and Google.

\subsubsection{How to solve it}

So, we must create desire. Whether that is a desire to know more, a desire to see 'what happens next' or a desire to avoid a negative outcome.
Avoiding a negative outcome innevitably involves invoking some fear. Before we cast this as some Machiavellian evil, consider that the simple
technique of using a skull-and-crossbones icon, or red text as a way of indicating danger or some situation that requires attention, is an
application of exactly this technique. We are invoking a fear-response, however mild.

Note however that not all triggers are culturally independent. Colour is particularly interesting in this respect, and your use of colour might
need to be modified for different target audiences (\textbf{NEED REFERENCE}).

\subsubsection{How we solve it. Aire Guru}

\elsparagraph{Evaluation}  
\begin{itemize}
    \done
    \crossed
    
\end{itemize}
\newpage