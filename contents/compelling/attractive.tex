\subsection{Attractive}
Es importante que la informacion resulte atractiva, llamativa, con chispa. No solo desde un punto de vista conceptual
o funcional como vimos en el apartado de relevante, donde dejamos claro la implicacion de los datos y la importancia
de su correcta presentacion, con una estructura que guie al usuario de una manera logica y ordenada.

Tiene que agarrar nuestra atencion, para ello podremos proponernos los siguientes requisitos, que sea divertida, entretenida,
moderna, actual, cercana, de calidad, bella y armonica.

Aunque estas caracteristicas sean subjetivas, existen tecnicas para poder lograr estos objetivos.

Otras caracteristicas mas tecnicas son:\\
 
\textbf{Consistencia.} Los controles que realizen las mismas acciones debe mantener el mismo diseno.\\

\textbf{Limpieza.} \\

\textbf{Simplicidad.} \\

\textbf{Familiaridad.} \\

\textbf{Estandares.} \\

\textbf{Detalles.} \\

Sena de identidad

\subsubsection{How to solve it} 
Para poder realizar un diseno atractivo, hay que estudiar las tendencias del momento y aplicarlas. 

Elementos de la vida real
respetar colores para las misma funciones
\subsubsection{How we solve it. Aire Guru} 
Como comentabamos anteriormente el diseno se ha basado en material design, utilizar lineas rectas, diferencia
distintas secciones mediante rectangulos. Estas secciones se componen de una cabecera y un cuerpo. Esta estructura
se repite durante todo el diseno.
Tipo de letra
Contraste
No hay muchos colores, combinacion reducida de tonalidades
Icono simple y con significado
Nombre, significado
Estandares, funcion de espera, cambio de color con pulsacion, subrayado para los links
Iconografia representativa
Efectos graficas se dibujan progesivamente
Diseno personalizado

\elsparagraph{Evaluation}  
\begin{itemize}
    \done
    \crossed
    \crossed Mas diseno
\end{itemize}

\begin{comment}
    Simplicidad,limpieza
Este concepto es tambien aplicable a la representacion, debe seguir el mismo principio, huiremos de decoraciones inecesarias que hagan al 
usuario perderse, solo debe contener la informacion necesaria para poder ser interpretada correctamente, es decir,
toda representacion tienen que contarnos lo que significa sin tener que estar accediendo continuamente a informacion
extra.
Colores consistentes

Estructuras consistentes

Moderna

Calidad

Cercania
Ademas se debera buscar la manera de presentar la informacion de una manera armonica y agradable para el usuario.
\end{comment}
\newpage