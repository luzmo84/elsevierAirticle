\subsection{Make sure your data is reliable}

It is essential that the credibility of your dataset be impeccable. The source of the data must be reliable.
Usually, in the description of the data, a section should be included where the origin of the dataset is explained.\\

The quality of the dataset can be measured according to the following characteristics of the values of the fields:

\begin{itemize}
    \item \textbf{Precision}. Units of measurement in the correct scale. For example, if we are measuring people's height, it 
    would be meaningless to provide these measurements in integral meters, since the possible values will be 0, 1 and 2. \\

    \item \textbf{Consistency}. Logical values for each type of field. An example is date field, which should all should
    follow the same format and use the same time zone. Don't confuse metric and imperial units!\\

    \item \textbf{Interpretability}. The data must be clear, both in values and in units. For example, we cannot
    suppose the unit or scale of the data.

    \item \textbf{Completeness}. The data must be consistent in each of the samples. The samples can be differentiated
    from each other, as we see in the Figure below, can share some fields and not others. It would be
    better to have a high degree of compliance for the data that interests us in our design.
\end{itemize}

\subsubsection*{Suggested strategies} 

\begin{itemize}
    \item Validate your data against its origin. With the dataset and the definition
    of each of the fields, the scale and interpretability of the fields and values must be checked.
    \item Check for extreme or strange values.
    \item Use database analysis tools to check the completeness of the fields.
\end{itemize}

\subsubsection*{In the context of Aire Guru \ldots} 

The source of data is verified by both the CEMI (Málaga information center) and UrbanClouds, the private company
that collects and sends the data to the city of Málaga. With the description of the data in the open data portal of Málaga and the information received by UrbanClouds,
we obtain all the necessary information to interpret the dataset. \\

For each field, a graphical study has been carried out with the Tableau tool and has verified that the values of the
fields meet the scale and range of expected values, that is, they are precise and consistent.
Using the analysis tools of MongoDBCompas and NoSQLBooster we verify the completeness of the fields needed for our model.

\begin{center}
    \bf{ 
    Figure 6.2.1. Completeness analysis}
\end{center} 