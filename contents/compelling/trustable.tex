\subsection{Trustable}
La confianza entre la plataforma que queramos utilizar y el usuario tiene que ser firme.
Por ello tendremos que garantizar la proteccion de su intimidad y darle la certeza que si 
utilizamos sus datos es solo exclusivamente para un fin en especifico y le dejaremos claro,
cual es ese fin, es decir, transparencia. 
Deberemos por ello minimizar el riesgo de la divulgacion de sus datos, es decir, guardar
por su seguridad, integridad y confidencialidad. 
Para ello deberemos seguir tanto las normas legales como una etica de buenas practicas.
No obligar a un identificado si se puede mostrar informacion.
Siempre pedir permiso para la captura de datos y dejar claro la finalidad de estos.


\subsubsection{How to solve it} 
Utilizacion de plataformas seguras, como puede ser SSL si desarrollamos una aplicacion web.
Si utilizamos API de terceros, asegurarnos tanto de su reputacion como el nivel de seguridad que ofrece.
Implementar un sistema que no permita relacionar los datos con un usuario.
Dar al usuario la posibilidad de eliminar sus datos si asi lo desea.


\subsubsection{How we solve it. Aire Guru} 
Plataforma segura SSL
Login con API firebase
Encriptacion del token firebase con token propio
La recoleccion de datos de posicion no es exhaustiva, pero garantiza la precision.
Pide permiso para leer la ubicacion
Deja que el usuario elimine los datos si lo desea.
Usuario debe realizar una accion para leer su localizacion.
A la hora de mostrar el historial personal, te explica para que sirve estar identificado y es opcional.

 --foto workflow identificacion
 --foto pedir permiso
 --foto eliminacion de datos
 --foto explicacion de para que se quieren los datos.
\elsparagraph{Evaluation}  
\begin{itemize}
    \done Codigo de buenas practicas
    \done No es posible relacionar la identificacion del usuario de firebase con nuestra base de datos
    \crossed Algunos usuarios nos indicaron que preferirian crear una cuenta con un nombre de usuario y contrasena
    y no utilizar su cuenta de correo por miedo al "robo de informacion"
    
\end{itemize}
\newpage