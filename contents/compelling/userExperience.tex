\subsection{User Experience}
 
--atractive
-workflow
-intuitive
Disponibilidad --En este caso no se necesitan aparatos que midan los datos
--diseno simple y minimalista
--workflow

Desde un punto de vista funcional se ha de tener en cuante, que para lograr el mayor alcance de usuarios, de distintos tipos, se han
de repetar las siguientes premisas,  la informacion debe ser presentada en un formato facil de entender por
el usuario, de una forma logica, que respete un order, es decir, de menos detalles a mas, que le de la posibilidad al usuario de indagar por si mismo, como
si la redaccion de una novela se tratara.
La representacion debe estar fuera de toda ambiguidad, no mostrar informacion que resulte enganosa o que haya que dedicar demasiado tiempo fijandonos
en los detalles para poder interpretarla correctamente.
Ademas se debera buscar la manera de presentar la informacion de una manera armonica y agradable para el usuario.
Al represnentar la informacion sera incluido lo basicamente necesario, huyendo siempre de graficas recargadas que solo aporten ruido a la represetacion.

\begin{itemize}
    \item facil de entender
    \item Analisis del publico al que va dirijido
    \item agradable
    \item no enganar
    \item nivel alto de detalle sin datos personales
    \item actual
    \item Herramientas necesarias para que el usuario entienda la informacion y de porque
    \item formato adecuado para un amplio espectro de usuarios
    \item Crear familiaridad
    \item Flexibilidad de seleccion de subdata
    \item No decoracion que ensucie o distraiga de la informacion principal
    \item Consecuente, que siga una secuencia logica
    \item Estudio de las representaciones mas adecuadas para cada tipo de datos
      \end{itemize}
    
\subsubsection{How to solve it} 


\subsubsection{How we solve it. Aire Guru} 
 
\elsparagraph{Evaluation}  
\begin{itemize}
    \done
    \crossed
    
\end{itemize}
\newpage