\subsection{User Experience}
 El sistema puede mostrar los datos de una forma precisa, optimizada, totalmente personaliza y ofrecerle
 la posibilidad al usuario de mejorar su vida considerablemente, sin embargo, si no proporcionamos una experiencia
 de usuario agradable y fluida, todos nuestros esfuerzos seran en vano.
 El usuario debe sentir que el sistema le ofrece la informacion de una manera fluida, limpia, intuitiva, directa y con
 una secuencia logica.
  
Esta secuencia le permitira al usuario la posibilidad de indagar por si mismo, como si la redaccion de una novela se tratara.

La representacion debe estar fuera de toda ambiguidad, no mostrar informacion que resulte enganosa o que haya que dedicar demasiado tiempo fijandonos
en los detalles para poder interpretarla correctamente.
Ademas se debera buscar la manera de presentar la informacion de una manera armonica y agradable para el usuario.
Al representar la informacion sera incluido lo basicamente necesario, huyendo siempre de graficas recargadas que solo aporten ruido a la represetacion.

Buscaremos crear familiaridad, que el usuario pueda enterder de una forma rapida las estructuras de los datos y 
le resulte facil de moverse con el minimo tiempo de aprendizaje posible.

Es importante darle al usuario un grado de flexibilidad, que sea capaz de realizar sus propias selecciones para poder ver
los datos en los que esta interesando.

Estudio de las representaciones mas adecuadas para cada tipo de datos.
Es importante trabajar en los tiempos y no hacer esperar al usuario
Es importante que una accion tenga una respuesta aunque esta sea de espera.
    
\subsubsection{How to solve it} 
Diseno actualizado y moderno, limpio, lejos de decoraciones innecesarias, debera seguir una logica.
Accion-reaccion 
Tiempo de aprendizaje minimo, familiaridad, intuitividad
Flexibilidad

\subsubsection{How we solve it. Aire Guru} 
Diseno material design, colores combinados no estridentes, simple
Workflow, del mapa a mas detalle
Repeticion de estructuras
respetar colores para las misma funciones
Filtros en el mapa
filtros en las graficas
Se utiliza el mismo estilo, colores e iconografia en todo el diseno para que el usuario se familierize rapidamente y pueda
prestar atencion al significado de los datos en vez de perderse en el diseno e intentar encontrar su significado. 
\elsparagraph{Evaluation}  
\begin{itemize}
    \done Estetica
    \done Logica de secuencia
    \done Familiaridad, repeticion de estructuras y colores que representan lo mismo
    \done Selecciones
    \crossed Tiempos de espera
    \crossed Mas diseno
    
\end{itemize}
\newpage