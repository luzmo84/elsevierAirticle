\section{References}

Braidot, N.P. (2009). Neuromarketing. Ediciones Gestión 2000.\\

Burton, B., Geishecker, L., Hostmann B, Friedman, T. y Newman, D. (2006).
Organizational Structure: Business Intelligence and Information Management.
Gartner.\\

Cano, J. L. (2007). Business intelligence: competir con información. Madrid:
ESADE Business School.\\

Chiasson, T.; Gregory, D. et al. (2014). A simple introduction to preparing and
visualizing information. Columbia, Missouri: Donald W. Reynolds Journalism
Institute and Infoactive.\\

Chiasson, T.; Gregory, D. et al. (2014). DATA + DESIGN A simple introduction to
preparing and visualizing information. Columbia, Missouri: Donald W. Reynolds
Journalism Institute and Infoactive. \\

Chakrabarti, S. (2003). Mining the Web: Discovering Knowledge from Hypertext Data.
California. Morgan Kaufmann.\\

Davenport, T., and Prusak, L. (2000). Working Knowledge: How Organizations Manage
What They Know. Massachuset: Harvard Business Review Press.\\

Davenport, T. H., Barth, P. y Bean, R. (2012). How big data is different. MIT Sloan
Management Review.\\

Debenham, J. (1998) Knowledge Engineering. Unifying Knowledge Base and Database
Design. Sydney: Springer.\\

Devlin, B. (1997). Data Warehouse: From Architecture to Implementation. Estados
Unidos: Addison-Wesley.\\

Eckerson, W. y White. C., (2003). Evaluating ETL and Data Integration Platforms.
TDWI Report Series.\\

Few, S. (2012). Show me the Numbers. Burlingame, California: Analytics Press.
Meirelles, I. (2013). Design for Information. Beverly, Massachusetts: Rockport
Publishers.\\

Han, J., Kamber, M. y Pei, J. (2006). Data Mining, Concepts and Techniques (2º
edición). Massachusets: Morgan Kaufmann.\\

Jarke, M., Jesusfeld, M.A., Quix, C., and Vassiliadis, P. (1998). Arquitecture and Quality
of Data Warehouses: An Extended Repository Approach. Advanced Information Systems
Engineering, Lecture Notes in Computer Science.\\

Kerin, R. (2014). Marketing. McGraw Hill. 

Kirk, A. (2012). Data Visualization: A sucessfull Design Process. Birmingham (UK):
Packt Publishing.

Lin, T. Y. (2002). Attribute transformation for data mining I: theoretical explorations.
International Journal of Intelligent Systems.\\

Markham, K. (2018). A Practical Guide to the General Data Protection Regulation (GDPR).Law Brief Publishing \\

Muñiz, R. (2014). Marketing en el siglo XXI. Centro de estudios financieros. \\

Murray, S. (2013). Interactive Data visualization for the Web. Massachusetts (EUA):
O’Reilly.\\

Parasurman, A. Zaithaml, V. A. y Berry, L. L. (1998). Servqual: a multiple-item scale for
mesauring consumer perceptions of service quality. Journal of Retailing.\\ 

Pettersson, R. (2013). Information Design 4- Graphic Design. Austria: International
Institute for Information Design (IID).\\

Redman, T.C. (1996). Data Quality for the Information Age, p. 245-267. Massachusetts:
Artech House, Inc.\\

Solomon, M.R. (2013). Comportamiento del consumidor. Addison-Wesley.\\

Telea, A. C. (2014). Data Visualization: Principles and practice, second Edition. Ohio
(EUA): CRC Press. \\

Tufte, E. R. (2001). The Visual Display of Quantitative Information. Cheshire,
Connecticut: Graphic Press.\\

Ware, C. (2013). Information Visualization. Waltham, Massachusetts: Morgan
Kaufmann Publishers. \\

Weller, K. (2010). Knowledge Representation in the Social Semantic Web. New York:
De Gruyter. \\


