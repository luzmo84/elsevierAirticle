\subsection*{}
Una buena representacion de los datos no consiste en presentarlos todos para que el usuario tenga la mayor informacion posible,
consiste en mostrarle los datos necesarios para que pueda hacer una lectura directa de la situacion.//

En este apartado veremos el refinado de los datos para ofrecer al usuario una informacion relevante y de calidad.

\subsection{Diseno}
Para realizar una buena seleccion de los datos que son relevantes o la combinacion de estos, es imprescindible unos 
conocimientos solidos sobre la materia a tratar. Por ello, es fundamental dedicarle el tiempo necesario para enter la
importancia de los datos y que beneficios nos puede aportar.

--cual es el objetivo?
--Como llegar a el?
--Que se debe enfatizar?
--Buscar puntos importantes
--Contar con ayuda si es necesario
--equipo multidisciplinar


\subsection{Calidad de los datos}

Una vez estipulados los campos que se requieren de la fuende de datos de origen, hay que serciorarse de que 
cada campo y valor, cumplen las siguientes premisas:
    \begin{itemize}
        \item \textbf{Credibilidad}. La fuente de datos debe ser fiable. Es decir, la fuente de origen debera
         proporcionar la informacion necesaria de como se han recogido los datos.
        \item \textbf{Precision}. Unidades de medida en la escala correcta. Por ejemplo, si estamos mediendo la altura
        de las personas, carecera de sentido proporcionar estas medidas en metros, ya que los posibles valores, seran 0, 1 y 2.
        \item \textbf{Consistencia}. Valores logicos para cada tipo de campo. Un ejemplo seria un campo fecha, todos deberian
        seguir el mismo formato.
        \item \textbf{Interpretabilidad}. Legible. 
        \item \textbf{Complitud}. Los datos deben ser consistentes en cada uno de las muestras. Las muestras pueden diferenciarse
        \item unas de otras, como vemos en la Figura 2, pueden compartir algunos campos y otros no. Para un buen resultado, seria 
        conveniente un grado de complitud alto para los datos que nos interesan en nuestro diseno.
    \end{itemize}

\begin{itemize}

    \item Informacion precisa
    \item Datos contrastados
\end{itemize}

\subsection{Util}

\subsection{Personalizacion}

\begin{itemize}

      \item Herramientas necesarias para que el usuario entienda la informacion y de porque
 
    \item fuentes fiables, actualizacion periodica
\end{itemize}