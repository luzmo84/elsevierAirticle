\subsection*{}
Una buena representacion de los datos no consiste en presentarlos todos para que el usuario tenga la mayor informacion posible,
consiste en mostrarle los datos necesarios para que pueda hacer una lectura directa de la situacion.\\

En este apartado veremos el refinado de los datos para ofrecer al usuario una informacion relevante y de calidad.

\subsection{Diseno}
Para realizar una buena seleccion de los datos que son relevantes o la combinacion de estos, es imprescindible unos 
conocimientos solidos sobre la materia a tratar. Por ello, es fundamental dedicarle el tiempo necesario para enter la
importancia de los datos y que beneficios nos puede aportar.

Los puntos que deben quedar claro en la fase de diseno son:

\begin{itemize}
    \item \textbf{Objetivo}. Conceptualmente, que problema queremos solventar o clarificar.
    \item \textbf{Puntos importantes}. Que limites o que valores de que datos son los que nos proporcinan una informacion
    de una situacion excepcional, o los valores que estamos intentando identificar mas facilmente.
    \item \textbf{Data workflow}. Diseno del flujo de informacion, que quede claro como se llega de un valor aparentemente
    insignificante a una informacion representativa.
\end{itemize}

Una vez marcada una meta, deberemos estipular cuales son los campos necesarios que necesitamos para representar la informacion
disenada. 
Es momento de realizar un analisis de los datos proporcionados y ver en la lista de los campos que nos ofrece la fuente, si serian
suficientes para alcanzar el objetivo marcado, si no directamente, con la combinacion de ellos.

Ademas es nuestra responsabilidad, asegurarnos de que cuentan con la suficiente calidad para poder usuarlos.

\subsection{Calidad de los datos}


Una vez estipulados los campos que se requieren de la fuende de datos de origen, hay que serciorarse en primer lugar de la credibilidad
de los datos, es decir, la fuente debe ser fiable, en la descripcion de los datos debe contar una seccion dende explique la procedencia
de los datos. Ademas, habra que comprobar las siguientes premisas para cada unos de los campos representados en los datos de origen:
    \begin{itemize}
        \item \textbf{Precision}. Unidades de medida en la escala correcta. Por ejemplo, si estamos mediendo la altura
        de las personas, carecera de sentido proporcionar estas medidas en metros, ya que los posibles valores, seran 0, 1 y 2.
        \item \textbf{Consistencia}. Valores logicos para cada tipo de campo. Un ejemplo seria un campo fecha, todos deberian
        seguir el mismo formato.
        \item \textbf{Interpretabilidad}. Legible. Los datos deber ser claros, tantos en valores como en unidades. Por ejemplo, no podemos
        suponer la unidad o la escala de los datos o, si los datos son dictonicos, "verdadero" o "falso", y los valores represetados son 
        "A" o "B", no podemos suponer verdadero= A y falso=B sin que esta informacion este descrita. 
        \item \textbf{Complitud}. Los datos deben ser consistentes en cada uno de las muestras. Las muestras pueden diferenciarse
        unas de otras, como vemos en la Figura  a continuacion, pueden compartir algunos campos y otros no. Para un buen resultado, seria 
        conveniente un grado de complitud alto para los datos que nos interesan en nuestro diseno.
    \end{itemize}

\begin{figure}[h]
\centering
\framebox{
\vbox{\begin{tabbing} 
Document \\
\hspace*{5mm} \= Sample 1 \\
\hspace*{10mm}\textit{Field A:} \hspace*{5mm} \= value A \\
\hspace*{10mm}\textit{Field B:} \hspace*{5mm} \= value B \\
\hspace*{5mm} \= Sample 2 \\
\hspace*{10mm}\textit{Field A:} \hspace*{5mm} \= value A \\
\hspace*{10mm}\textit{Field C:} \hspace*{5mm} \= value C \\
\hspace*{10mm}\textit{Field D:} \hspace*{5mm} \= value D \\
\hspace*{5mm} \= ... 
\hspace*{5mm} \= Sample n \\
\hspace*{10mm} \= ... 
\end{tabbing}}%
}
\caption{Ejemplo documento de la fuente de origen}
\end{figure}
\begin{comment}
    
\end{co}
\begin{itemize}

    \item Informacion precisa
    \item Datos contrastados
\end{itemize}
\end{comment}


\subsection{Util}
Como mencionamos anteriormente, huiremos de toda aquella informacion vana, que no haga ninguna aportacion
al objetivo marcado. No queremos que el usuario se distraga o pierda en medio de una piscina de datos completa
pero compleja que para el no tenga ningun sentido.\\

Los datos deben contarnos por si, que significan.
En caso complejos, los conceptos que no sean de uso diario para los usuario, podremos poner informacion
complementaria a su disposicion para que el usuario pueda informarse sobre el significado de los datos 
representados, esta practica proporcionara al usuario un mayor dominio del concepto y podra estar mas
seguro de su lectura de los datos.
Por otro lado, cuando dirigimos al usuario a fuentes oficiales de informacion, donde puedan obtener una explicacion veridica,
reenforzamos la credibilidad de las representaciones que estamos realizando.

\subsection{Personalizacion}
\begin{comment}
--relevante de manera general vs relevante personal
--casos personales
--aplicabilidad
--solventar problema en particular

    

\begin{itemize}

      \item Herramientas necesarias para que el usuario entienda la informacion y de porque
 \item conocimientos multidisciplinares
    \item fuentes fiables, actualizacion periodica
\end{itemize}
\end{comment}