\subsection{Calidad de los datos}
 
Una vez estipulados los campos que se requieren de la fuende de datos de origen, hay que serciorarse en primer lugar de la credibilidad
de los datos, es decir, la fuente debe ser fiable, en la descripcion de los datos debe contar una seccion dende explique la procedencia
de los datos. Ademas, habra que comprobar las siguientes premisas para cada unos de los campos representados en los datos de origen:
    \begin{itemize}
        \item \textbf{Precision}. Unidades de medida en la escala correcta. Por ejemplo, si estamos mediendo la altura
        de las personas, carecera de sentido proporcionar estas medidas en metros, ya que los posibles valores, seran 0, 1 y 2.
        \item \textbf{Consistencia}. Valores logicos para cada tipo de campo. Un ejemplo seria un campo fecha, todos deberian
        seguir el mismo formato.
        \item \textbf{Interpretabilidad}. Legible. Los datos deber ser claros, tantos en valores como en unidades. Por ejemplo, no podemos
        suponer la unidad o la escala de los datos o, si los datos son dictonicos, "verdadero" o "falso", y los valores represetados son 
        "A" o "B", no podemos suponer verdadero= A y falso=B sin que esta informacion este descrita. 
        \item \textbf{Complitud}. Los datos deben ser consistentes en cada uno de las muestras. Las muestras pueden diferenciarse
        unas de otras, como vemos en la Figura  a continuacion, pueden compartir algunos campos y otros no. Para un buen resultado, seria 
        conveniente un grado de complitud alto para los datos que nos interesan en nuestro diseno.
    \end{itemize}

\begin{figure}[ht]
\centering
\framebox{
\vbox{\begin{tabbing} 
Document \\
\hspace*{5mm} \= Sample 1 \\
\hspace*{10mm}\textit{Field A:} \hspace*{5mm} \= value A \\
\hspace*{10mm}\textit{Field B:} \hspace*{5mm} \= value B \\
\hspace*{5mm} \= Sample 2 \\
\hspace*{10mm}\textit{Field A:} \hspace*{5mm} \= value A \\
\hspace*{10mm}\textit{Field C:} \hspace*{5mm} \= value C \\
\hspace*{10mm}\textit{Field D:} \hspace*{5mm} \= value D \\
\hspace*{5mm} \= ... 
\hspace*{5mm} \= Sample n \\
\hspace*{10mm} \= ... 
\end{tabbing}}%
}
\caption{Ejemplo documento de la fuente de origen}
\end{figure}
\begin{comment}
    
\end{co}
\begin{itemize}

    \item Informacion precisa
    \item Datos contrastados
\end{itemize}
\end{comment}


\subsubsection{How to solve it} 


\subsubsection{How we solve it. Aire Guru} 
 
\elsparagraph{Evaluation}  
\begin{itemize}
    \done
    \crossed
    
\end{itemize}
 

\newpage