\subsection{Diseno}
Para realizar una buena seleccion de los datos que son relevantes o la combinacion de estos, es imprescindible unos 
conocimientos solidos sobre la materia a tratar. Por ello, es fundamental dedicarle el tiempo necesario para enter la
importancia de los datos, que beneficios nos puede aportar o que datos nos perjudican.

Los puntos que deben quedar claro en la fase de diseno son:

\begin{itemize}
    \item \textbf{Objetivo}. Conceptualmente, que problema queremos solventar o clarificar. Es muy importante que tengamos un objetivo claro,
    incluso si no se cuentan con conocimientos previos de la materia, tener un objetivo bien definido nos ayuda a encontrar la solucion o 
    describirla con mas claridad al experto.
    \item \textbf{Puntos importantes. Valores destacables}. Como mencionamos anteriormente, nuestro objetivo no es mostrar todos los datos, sino
    que se trata de proporcionarle los datos importantes al usuario para poder descubrir los valores significativos, los que tienen influencia en
    nosotros, ya sea positivo o negativo. A nivel mas tecnico, serian los limites o que valores de que datos son los que nos proporcinan una informacion
    de una situacion excepcional, o los valores que estamos intentando identificar mas facilmente.
    \item \textbf{Data workflow}. Diseno del flujo de informacion, que quede claro como se llega de un punto a otros, por ejemplo de una representacion
    simple a mas detallada o compleja. Si el usuario entiende los conceptos basicos, le sera mas facil de entender el concepto mas desarrollado.
\end{itemize}

Respecto al conjunto de datos, una vez marcada una meta, deberemos estipular cuales son los campos necesarios del conjunto de datos que necesitamos para representar la informacion
disenada. Es momento de realizar un analisis de los datos proporcionados y ver en la lista de los campos que nos ofrece la fuente, si serian
suficientes para alcanzar el objetivo marcado, si no directamente con sus valores, con la combinacion de ellos.

\subsubsection{How to solve it} 
Marcarnos un objetivo, estipulamos los campos necesarios, seleccionamos estos campos en el conjunto de datos. 
Estipulamos los valores destacables que les proporcionen mas informacion al usuario como las Situaciones excepcionales, o lo que estamos buscando.

\subsubsection{How we solve it. Aire Guru} 
Nuestro objetivo principal es crear el concienciamiento de la importancia de la polucion del aire en nuestra salud. Para ello investigamos sobre que
medida representa el nivel de polucion en el aire, en este caso es el AQI. Investigamos sobre ello en fuentes oficiales, para la ciudad de Malaga se aplica la 
legislacion europea, por lo que nos centramos en esta normativa.
Buscamos informacion de como se cualcula y cuales son los niveles nocivos.
Por otro lado tratamos de relacionar que enfermedades son afectadas por que contaminante y en que nivel les afecta. Ademas de las enfermedades, estudiamos sus 
sintomas y como se pueden ver afectados.
Para cada contaminante, buscamos las fuentes de polucion y vemos si se aplican a la ciudad de Malaga.
Con toda esta informacion definimos que los contaminantes que afectan a la calidad del aire son el CO, NO2, O3 y PM (Particulated Material), diferentes medidas de particulas
afectan de distinto modo a distintas enfermedades.

Una vez seleccionados estos contaminantes, verificamos si contamos con estos datos en nuestro conjunto de datos. Como mencionamos anteriormente algunas muestras
muestran medidas cualitativas y otras no cuentan con las medidas. En este caso se deciden los siguientes valores Bueno:0, ..... y se le indicara al usuario cuando el valor
es cualitativo.


Respecto al enfasi en los valores ....


---como es de concienciacion, ofrece la informacion

Investigacion del AQI, investigacion enfermedades y sintomas, fuentes de polucion y como afecta cada contaminante.
 Color rojo, iconografia, barras, altura facil de interpretar por el usuario.
\elsparagraph{Evaluation}  
\begin{itemize}
    \done
    \crossed
    
\end{itemize}
 

\newpage