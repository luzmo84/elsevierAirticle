\subsection{Diseno}
Para realizar una buena seleccion de los datos que son relevantes o la combinacion de estos, es imprescindible unos 
conocimientos solidos sobre la materia a tratar. Por ello, es fundamental dedicarle el tiempo necesario para enter la
importancia de los datos y que beneficios nos puede aportar.

Los puntos que deben quedar claro en la fase de diseno son:

\begin{itemize}
    \item \textbf{Objetivo}. Conceptualmente, que problema queremos solventar o clarificar.
    \item \textbf{Puntos importantes}. Que limites o que valores de que datos son los que nos proporcinan una informacion
    de una situacion excepcional, o los valores que estamos intentando identificar mas facilmente.
    \item \textbf{Data workflow}. Diseno del flujo de informacion, que quede claro como se llega de un valor aparentemente
    insignificante a una informacion representativa.
\end{itemize}

Una vez marcada una meta, deberemos estipular cuales son los campos necesarios que necesitamos para representar la informacion
disenada. 
Es momento de realizar un analisis de los datos proporcionados y ver en la lista de los campos que nos ofrece la fuente, si serian
suficientes para alcanzar el objetivo marcado, si no directamente, con la combinacion de ellos.

Ademas es nuestra responsabilidad, asegurarnos de que cuentan con la suficiente calidad para poder usuarlos.

\subsubsection{How to solve it} 


\subsubsection{How we solve it. Aire Guru} 
 
\elsparagraph{Evaluation}  
\begin{itemize}
    \done
    \crossed
    
\end{itemize}
 

\newpage