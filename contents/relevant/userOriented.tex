\subsection{Enfoque al usuario}
Una gran parte del objetivo marcado en el apartado 3.1 Diseno, viene definido por el servicio que queramos ofrecerle
al usuario, ya que este, es nuestro cliente final y al que pretendermos ayudar.
Tendremos que definir las necesidades que muestran los usuarios en combinacion con las carencias del mercado.
Para ello tendremos que decidir si nuestro objetivo esta dirigido a un publico de gran dimension, por lo que tendremos que
que identificar las necesidades homogeneas del usuario medio o si el objetivo esta enfocado a un tipo de usuario
especifico, con necesidades caracteristicas.

Por lo tanto necesitamos un estudio actual sobre el nivel psicologico, social y economico del grupo objetivo, para descubrir, no solo
las necesidades, pero tambien los deseos, que le gustaria tener y sus demandas, que quieren.

Una vez los usuarios tengan acceso a los datos tal y como se los proporcionamos, es importante tener una retroalimentacion, 
seria ideal poder realizar un analisis del comportamiento del usuario para ve sus pautas respecto a los datos obtenidos.

\subsubsection{How to solve it} 


\subsubsection{How we solve it. Aire Guru} 
 
\elsparagraph{Evaluation}  
\begin{itemize}
    \done
    \crossed
    
\end{itemize}
 

\newpage