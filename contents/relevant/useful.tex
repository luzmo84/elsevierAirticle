\subsection{Util}
--Que es informcion no util
--No perdernos en definiciones o explicaciones que no son nuestra especialidad.
-No porque este en el conjunto de datos significa que hay que ponerlo.
Como mencionamos anteriormente, huiremos de toda aquella informacion vana, que no haga ninguna aportacion
al objetivo marcado. No queremos que el usuario se distraga o se pierda en medio de una piscina de datos completa
pero compleja que para el no tenga ningun sentido.

Toda representacion tienen que contarnos lo que significa sin tener que estar accediendo continuamente a informacion
extra.
En caso complejos, los conceptos que no sean de uso diario para los usuario, podremos poner informacion
complementaria a su disposicion para que el usuario pueda informarse sobre el significado de los datos 
representados, esta practica proporcionara al usuario un mayor dominio del concepto y podra estar mas
seguro de su lectura de los datos.
Por otro lado, cuando dirigimos al usuario a fuentes oficiales de informacion, donde puedan obtener una explicacion veridica,
reenforzamos la credibilidad de las representaciones que estamos realizando.
--huir de decoraciones

\subsubsection{How to solve it} 


\subsubsection{How we solve it. Aire Guru} 
 Nos centramos en 6 contaminantes y la medida es unica para todos y el general, el AQI, los rangos son exactamente
 los mismos para todos
 No existen datos que el usuario no entienda, como coordenadas, identificador de la estacion, como se ha tomado la medida.
 Si la medida no existe, simplemente la obviaremos.

\elsparagraph{Evaluation}  
\begin{itemize}
    \done
    \crossed
    
\end{itemize}
 \newpage