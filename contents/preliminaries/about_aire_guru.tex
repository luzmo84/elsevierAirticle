\subsection{About Aire Guru}
Trust between the platform we want to use and the user is essential.
Therefore we will have to guarantee the protection of user privacy and promise that user data will be used only for specific purposes that the user clearly agrees to.
We must minimize the risk of disclosing user data, and maintain user safety, confidentiality and integrity of information.
For this we must follow both the legal norms and an ethics of good practices.
Do not require user identification merely to show general information.
Always ask for permission to capture data and make clear what it will be used for.

\subsubsection*{How to solve it} 
Use secure platforms, such as SSL if we develop a web application.
If we use third-party APIs, make sure to check the level of security they offer.
Implement a system that does not allow linking the data with a user.
Give the user the possibility to delete their data if they wish.


\subsubsection*{How we solve it. Aire Guru} 
In our case, security is high, since for our main purpose, personal history
of exposure to contamination, it is not essential to collect the user's details.
SSL has been used to achieve a level of competent security, which guarantees the encryption of
the information sent through the network.
For the identification of the user a secure and proven API has been used. We use the identification
that Firebase provides. This API provides us with an identifier, which may be encrypted.
We have of course used encryption. When the data reaches our database,
we check the user and then we re-encrypt the user with our own salt to
avoid linking firebase with our database.

We never store the users exact location, only the polygon where he is.
We have also implemented a time limit, which allows us to have the necessary precision to show exposure to pollutans, but doen't allow user tracking.

As per the code of good practice, never collect the user's position without their permission.
To get this permission, one should require an explicit action. In our case, the user will have to select "Mi ubicacion" (My location) and accept 
that we may determine their position. Of course the user can always revoke this permission and
delete their data at any time. \\

\begin{figure}[ht]
    \centering
   \subfigure[Location Access]
    {\includegraphics[width=5.5cm  ]{locationAccess}}
    \hfill
    \subfigure [Settings]
       { \includegraphics[width=5.5cm]{settings}}
   
  \caption{Filters}
    \end{figure}

    The collection of locational data is not exhaustive, but guarantees accuracy. \\

    As we want a transparent tool, we explain to the user that we need their data before
    they are identified, and of course, the identification is optional. \\

    \begin{figure}[ht]
        \centering
        \includegraphics[width=8cm]{loginInfo}
        \caption{Info before login}
    \end{figure}

\elsparagraph{Evaluation}  
\begin{itemize}
    \done Code of good practices
    \done It is not possible to relate the identification of the firebase user with our database
    \crossed Some users indicated that they would prefer to create an account with a username and password
         rather than use their email accounts, for fear of "information theft"
    
\end{itemize}