\newpage
\section{Introduction}

Making scientific and technical information available to the general public has always been a challenge.
Often such information, the use of which could be of enormous benefit to the public, falls on deaf ears.\\

We now have access to a new form of scientific data - 'big data' - massive numbers of data points.
This data is accumulating at enormous rates, and their is an obvious desire to use it, both for
commercial and 'public good' reasons.\\

We have ways of delivering this information that are effectively free, and can be made extremely compelling
if presented in the right way. This chapter is a guide to 'the right way'.\\

We have created a tool - Aire Guru - which collects air pollution data periodically, stores it for historical use, 
processes it to extract relevant information, personalizes it to engage the user with regard to their individual
circumstance, visualizes it in an easy to understand format, and provides it online to facilitate its accessibility. Furthermore,
descriptions of all the data and how it has been processed are also provided to raise public awareness of the value
of this data.\\

While developing Aire Guru, and later reflecting on the process and success of the project, we we motivated to codify
some of the lessons learnt in bringing somewhat dry and technical data to the general public, in a way that made it 
accessible, relevent, and compelling.\\

\subsection{The structure of this document}

We first provide a short introduction to the Aire Guru project in order to provide context.
In the following sections, we provide a set of guidelines that can be used to make big data
accessible, relevant and compelling for the average user. Such principles are applicable to 
online products in general, but we consider them here with special reference to presenting
data, as opposed to e.g. social media sites. \\

Firstly, we deal with general preliminaries, that in a commercial context would be called
'Market Research' and 'The Business Case'. These ideas are just as applicable to the delivery of
public-good data as commercial offerings. \\

We then present 14 specific principles, classified into three categories: Accessible, Relevent, and Compelling.
Each principle is described, along with suggested strategies to achieve that goal, and we then show
how that principle is reflected in the context of our Aire Guru project. We conclude with some
general comments, and provide a suggested reading list for each major section. \\

We have applied these same principles to the design of this chapter.