\newpage
\section{Introduction}

Making scientific and technical information available to the general public has always been a challenge.
Often such information, the use of which could be of enormous benefit to the public, falls on deaf ears.\\

We now have access to a new form of scientific data - 'big data' - massive numbers of data points.
This data is accumulating at enormous rates, and their is an obvious desire to use it, both for
commercial and 'public good' reasons.\\

We have ways of delivering this information that are effectively free, and can be made extremely compelling
if presented in the right way. This chapter is a guide to 'the right way'.\\

We have created a tool - Aire Guru - which collects air pollution data periodically, stores it for historical use, 
processes it to extract relevant information, personalizes it to engage the user with regard to their individual
circumstance, visualizes it in an easy to understand format, and provides it online to facilitate its accessibility. Furthermore,
descriptions of all the data and how it has been processed are also provided to raise public awareness of the value
of this data.\\

While developing Aire Guru, and later reflecting on the process and success of the project, we we motivated to codify
some of the lessons learnt in bringing somewhat dry and technical data to the general public, in a way that made it 
accessible, relevent, and compelling.\\

\subsection*{The structure of this document}

In the following sections, a tour through the necessary steps to make this data
accessible, relevant and compelling for the average user.
The objective of this chapter is to stipulate the necessary points to use the data in a profitable manner and
to help us in our day to day in a useful way. These concepts have been applied in the Air Guru project,
that brings the data to users in an easy, friendly, compressible and adapted to their needs. \\

In this chapter we will see how to make the data accessible, relevant and compelling for 
the average user. For each one of the secluded ones, we will see the theory, as it has been applied
in the Air Guru project and its evaluation.
Next, we will see a general evaluation of these three concepts and to finish the 
obtained conclusions. \\

This chapter is written with the objectives we want to achieve in mind for the reusing of bigData. It means, make the 
concepts accessible, relevant and compelling. It intents to be a guide which show these principles
in its structure.
