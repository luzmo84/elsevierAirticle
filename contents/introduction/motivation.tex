\subsection*{Motivation}
Con el paso de los annos hemos visto en el procesado de los datos una oportunidad de oro, 
tanto para mejora nuestra vida diaria como para intentar predecir
eventos futuros. Por esta razon, surge la tendecia de almacenar todos los datos que tenemos 
a nuestro alrededor con la esperanza de que algun dia sean tratados.
Como por el momento no tienen ningun fin en concreto, se almacena competamente todo, sin discriminacion.
Sin saber si en algun momento seran utiles o no.\\

Por otra parte, multiples empresas tanto publicas como privadas, incluidas nuestros gobiernos, 
en su compromiso por la transparecia y/o esperanza de sacarle provecho, publican estos datos 
periodicamente.\\

Aunque estos datos esten disponibles para todos los usuarios, no significa que sean provechosos para el 
usuario medio, ya que se enfrenta a multiples retos.\\

En los siguientes apartados se realizar un recorrido por las etapas necesarias para hacer estos datos
accesibles, relevantes y atractivos para el usuario medio.
El objetivo de este capitulo es estipular los puntos necesarios para usar los datos de una manera provechosa y
que nos ayude en nuestro dia a dia de una forma util. Estos conceptos se han aplicado en el proyecto Aire Guru, 
que acerca los datos a los usuarios de una manera facil, amigable, compresible y adaptados a sus necesidades.\\

Aire Guru es una herramienta web para la monitorizacion personalizada de la calidad del aire en Malaga.


