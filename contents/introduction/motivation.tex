\subsection*{Motivation}

With the passage of the years we have seen in the processing of the data a golden opportunity, 
both to improve our daily life and to try to predict
future events. For this reason, the tendency arises to store all the data we have
around us with the hope that one day they will be treated.
As for the moment they have no specific purpose, everything is stored completely, without discrimination.
Without knowing if at some point they will be useful or not. \\

On the other hand, multiple companies, both public and private, including our governments, 
in their commitment to transparency and / or hope to take advantage of it, they publish this data 
periodically \\

Although these data are available to all users, it does not mean that they are useful for the 
average user, since he faces multiple challenges. \\

In the following sections, a tour through the necessary steps to make this data
accessible, relevant and compelling for the average user.
The objective of this chapter is to stipulate the necessary points to use the data in a profitable manner and
to help us in our day to day in a useful way. These concepts have been applied in the Air Guru project,
that brings the data to users in an easy, friendly, compressible and adapted to their needs. \\

Aire Guru is a web tool for personalized monitoring of air quality in Malaga.
