\subsection*{}
Con el paso de los annos hemos visto en el procesado de los datos una oportunidad de oro, tanto para mejora nuestra vida diaria como para intentar predecir
eventos futuros. Por esta razon, surge la tendecia de almacenar todos los datos que tenemos a nuestro alrededor con la esperanza de que algun dia sean tratados.
Como por el momento no tienen ningun fin en concreto, se almacena competamente todo, sin discriminacion.
Por otra parte, multiples empresas tanto publicas como privadas, incluidas nuestros gobiernos, en su compromiso por la transparecia, publican estos datos periodicamente.
Sin embargo, el usuario medio, no tiene ni los conocimientos, herramientas e infraestructuras para extraer informacion de estos datos.

--Datos automatizados -- formato maquina
--almacenamiento
--conocimientos informaticos
--analiticos
--estudio de la materia, ver que es importante o no, contrastarlo
--equipo multidisciplinario
--informaticos no son expertos en medicina, por ejemplo.
\begin{itemize}

    \item datos disponible
    \item no substraible por el usuario medio
    \item imposible de tratar sin una estructura
    \item no es posible de tener un historico
    \item sin informacion complementaria que muestre informacion util
    \item Desperdicio
    \item \end{itemize}
    