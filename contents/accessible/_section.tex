\section{Accessible}

The data is considered accessible by the user when they do not need to make an excessive effort to
to obtain it, understand it and use it, that is, apply them to their particular situation. To achieve these objectives,
it is necessary that the user can access them in a way that is familiar and in a language and nomenclature
they can understand.\\ 

In this section we consider five questions related to making sure your data is physically and psychologically accessible:

\begin{itemize}
    \item Is your data easy to obtain?
    \item Are you putting up psychological barriers?
    \item Is your data too low-level?
    \item Does your data required too much expertise or use specialist terminology?
    \item How do potential users find out that you exist?
\end{itemize}

\subsection{Make your data available}
    
As we mentioned in the introduction to this chapter, the datasets may be available in the original source. However, this doesn't necessarily
mean that they are readily accessible.
The challenges we encounter with the raw data direct from the source of origin, are the following: \\   
 
\textbf{Location}. Datasets are usually available in open data portals that have been organized and structured in particular ways.
Despite more companies attempting to offer functional and efficient user interfaces, often a complicated search and selection process is necessary for a user to find what they need. Sometimes the data may not all be available in the one location, but may require searching through multible portals.

\textbf{Extraction}. Datasets are usually available through an API (Application Programming Interface). APIs are not easily interpretable by the average user. Normally there will be a document describing the fields and values presented, and indicating how to actually use the API. \\

\textbf{Readability}. Datasets are usually represented in a format designed to be processed by software. This is fairly unintelligible to a human user. At best,
the data will be represented in a table and even then, it will often be quite difficult to extract the required information.

Therefore, we can not say that data interfaces are commonly accessible in a useful way for the average user. \\

\subsubsection*{Suggested strategies} 

We must provide the information required by the user in an uncomplicated manner, in a format that is easy to read and interpret, and we must describe it in a way that is easy to understand and quick to digest..\\

In order to collect the information which is relevant to the user, we will need to carry out processes such as extraction, transformation and
data cleaning.
 
\subsubsection*{In the context of Aire Guru \ldots} 

Our webtool uses the air quality data provided by the city of Málaga in its open data portal.(https://datosabiertos.malaga.eu/)\\

\begin{figure}[ht]
    \centering
    \subfigure[Main page]
        {\includegraphics[width=5.5cm]{openDataPortal}}
    \hfill
    \subfigure [Category environment]
        {\includegraphics[width=5.5cm]{openDataPortalEnviromentCategory}}
    \vfill
    \subfigure[GeoJson Document]
        {\centering \includegraphics[width=4.75cm]{geoJsonAirQualityDataRaw}}
    \caption{Open Data Portal Málaga}
\end{figure}

This data portal offers a variety of categories (represented by different icons) indicating classifications of the dataset.
Once a category is selected, the user is presented with a search bar that allows them to search for specific data using keywords.\\

In this case, if we click on the link, data is displayed in a new tab. The use of software
for translating the data is not strictly necessary, but we can see that the format (JSON) is not easily readable in human terms.

Aire Guru offers all the necessary information on a web platform which has been designed to facilitate human understanding.

\begin{itemize}
    \item Location. Finding data is made simple, since information is displayed immediately on accessing the website. The main page
          presents the levels of pollution in all areas without the need to make any selection.
    \item Extraction. No specialist software or computer knowledge is necessary to access the information.
    \item Readability. Aire Guru includes a map which shows pollution levels, represented by different colors. These colors are defined by a legend
          below the map. It also includes a glossary explains the concepts presented on the website, the meaning of each section and includes clear instructions on how to
          use and navigate through the web page.
\end{itemize}
\subsection{Remove all barriers to access}
We already told about offering the data directly to the user but going back to the concept that available
does not mean accessible. If we lock ourself and make a completly private platform, we would lose the oportunity
to bring the information to the users.\\

We break these barriers building a service free of charge, or at least as free as it can be. Of course, if we are 
working with data of third parties and they have a restricted policy, we will need to adjust the rates to cover
these restrictions.\\

It is recomendable to do not force our users to identify theirself or provide data if this is not
extrictly necesary for the proper behaviour of our data representation. As we mention before, collecting data from our
users requires a major efford to ensure their security. Furthemore, we see like every year, the laws ajust to provent
the abuse with the data of the users. If we use any user's data, we should make sure it not compromise the law.\\

All of us have noticed how annoying is to be reading an article when more than the 50\% of the information is advertising.
We problably do not want to give the same impression to our users since it can repel them to the point that they will no longer
user our tool. If we need to use this kind of tecnics to support our system, We will problably choose the right places to offer 
these advertising without interfering in the representation of our information.


\subsubsection*{Suggested strategies} 

To remove the biggest amount of barriers we will try to offer a low rate system, no force the user to provide any unjustified 
data and avoid advertising which obstructs the representation of our data.

\subsubsection*{In the context of Aire Guru \ldots} 

Aire Guru is available online completely free of charge. It requieres login, since it shows the personal records of the 
user exposure to pollution. However, all the rest of the funtionabilities are offered to all the users, identified and no 
identified.

Aire Guru does not uses publicity.

\subsection{Don't just publish the raw data}

Recall that the main objective of accessing the data is to obtain knowledge, to understand and make sense of it.
Of course, pure data by itself has no value until we can understand it and apply it.
We must contextualize, process, and analyze the information to gain useful knowledge. \\ 
    
\begin{figure}[ht]
    \centering
    \includegraphics[width=12cm]{Figure_4_3_1_fromDataToInformationDiagram}
    \caption{From Data to Information Diagram}
 \end{figure}
 
\begin{center}
    \bf{        
    Figure 4.3.1. From Data to Information Diagram}
\end{center}
 
In order to obtain this knowledge, the user must correctly interpret the data.
It's not enough to know what the values and units individually represent, but what they mean in the big picture.
For that the user must already have expertise in the area, or must engage in further research allowing them to understand the data that has been extracted.\\
    
In order to build a system that makes data accessible, we need to transform the raw data into a derived model that our users can understand.
For this it will be necessary to have a solid knowledge of the dataset that is needed, the values, their units and how they relate to each other.
The platforms provide a huge volume of data, since, probably they will provide multiple samples containing different sets of fields. 
These field sets may be similar to each other but don't need to be identical.
From this data, we need to select the relevant fields necessary to represent specific model.\\

\subsubsection*{Suggested strategies} 

\begin{itemize}
    \item Design the derived model that can be understood by your users. You may need to use outside expertise to acquire the necessary knowledge on the subject.

    \item Transform the raw data into the derived model, using a series of processes such as extraction, transformation and cleaning of the data.
        Without automation, these processes are tedious and time consuming, so it is strongly advisable for the entire process to be automated.
\end{itemize}

\subsubsection*{In the context of Aire Guru \ldots} 

Aire Guru aims to increase the awareness of the level of pollution that surrounds us.
To do this, it uses a measure called the air quality index (AQI), specifically the European air quality index (EAQI).

\begin{figure}[ht]
    \centering
    \includegraphics[width=10cm]{Figure_4_3_2_aireGuru_landingPage_topSection}
    \caption{Aire Guru. Landing page. Top section}
\end{figure}
\begin{center}
    \bf{        
    Figure 4.3.2. Aire Guru. Landing page. Top section}
\end{center}

Aire Guru shows the AQI values for the city of Málaga, organized into zones, showing the levels of both the overall pollution and individual pollutants, from September 2018 to the present.
It also shows the evolution of these values in selectable timescales of days, months and years.

It is capable of creating a set of the most relevant pollutants for a given medical condition.
An innovative feature is the capacity to display levels of any particular pollutant by hour, day, month, or year. \\

The original source data is in GeoJSON format, a format which provides a JSON object with nested subdocuments.
Each of these subdocuments contains data in key-value form.
In the following figure we can see the beginning of the document downloaded on June 9, 2019 (https://datosabiertos.malaga.eu/recursos/ambiente/calidadaire/calidadaire.json) \\

\begin{figure}[ht]
    \centering
    \subfigure[First subdocument]
        {\centering \includegraphics[width=4.75cm]{Figure_4_3_3_a_geoJsonAirQualityData1}}
    \hfill
    \subfigure[Second subdocument]
        {\centering \includegraphics[width=4.75cm]{Figure_4_3_3_b_geoJsonAirQualityData2}}
    \caption{Air quality Document [09/06/2019]. Open Data Portal Málaga}
\end{figure}
    
\begin{center}
    \bf{        (a) First subdocument (b)Second subdocument \\
    Figure 4.3.3. Air quality Document [09/06/2019]. Open Data Portal Málaga}
\end{center}

In the Figure 4.3.3, We can see an excerpt of the first two subdocuments.
Each subdocument contains the coordinates of the air quality measuring station, the date and time when the measurement was recorded, and the values of the measurements.
In the Figure 4.3.4 we can find the description provided by the open data portal.\\
    
\begin{figure}[ht]
    \centering
    \includegraphics[width=8cm]{Figure_4_3_4_geoJsonAirQualityDataDescription}
    \caption{Air quality data description [09/06/2019]. Open Data Portal Málaga}
\end{figure}
\begin{center}
    \bf{        
    Figure 4.3.4. Air quality data description [09/06/2019]. Open Data Portal Málaga}
\end{center}

For a more detailed description of the measures, we have to resort to an external resource.
In this case we directly contacted the company that installs the UrbanClouds (https://urbanclouds.city/es/) measuring stations and provides the data to the Málaga city council.
After selecting the necessary fields according to our design plan, we carried out sequence of cleaning, transformation and extraction tasks:

\begin{itemize}
    \item \textbf{Cleaning}. We need to eliminate the repeated or non-relevant fields.
        For example, the identifier of the measuring station is unneccessary as the data already contains the coordinates of the station, and coordinate representation is more interesting for our purposes.

    \item \textbf{Transformation}. We need the values to have a format appropriate to the fields that they represent.
        For example, the date and time of the measurement is stored in date format instead of the string provided in the raw dataset.

    \item \textbf{Extraction}. We need to select the relevant fields.
        This dataset offers one or more measurements for each pollutant, which can be represented by three different fields:
        a quantitative measurement, a qualitative of the fixed station of measurement, and a qualitative station of a mobile station.
        We synthesize a field containing the measurement which is most relevant for our purposes, and eliminate the non-relevant values to minimize processing time.
\end{itemize}

For reliability purposes, a second, totally independent process collects and stores the raw data.
\subsection{Use concepts and language that your user understands}
The key point is that the user must be able to understand what we are trying to transmit. We must create a fluid, 
easy interaction between the user and the representation of the data. Therefore we must speak with the language of 
the ordinary person, using common vocabulary, avoiding specialist jargon and communicating the objective in the simplest, 
clearest manner. This representation of information plays a fundamentally important role, enabling the user to absorb 
the meaning in a natural way.

\subsubsection*{How to solve it} 
Consider what type of representational format is most appropriate to the task. If it is not possible, we will provide de 
resources needed to help the user to understand and place the information in context. 

We should stady which type of graphs fits better to which kind of data,A graph may not necessarily be the 
clearest way to represent information.

We should take into account the target audience, and how we can most effectively transmit information to them. 
If we decide that a using graph is the best solution, we must consider carefully which type to use. For example, 
if we talk about samples and we want to know the density, we will lean towards a density graph, and if we look for the difference
between sexes, we will use a pie chart.

\subsubsection*{How we solve it. Aire Guru} 
The Aire Guru tool presents the information in the native language of the city, using simple vocabulary and a 
straightforward style.
Colors and graphic resources are used as icons. In addition, a unified design structure to represent data has 
been used over the whole site, giving the user a consistent visualization experience. 

One of the objectives is to represent pollution by areas of the city, which is why a map has been used - the 
obvious and familiar visual image of the different places in the city. This is a more legible format since the user
does not need to make a continous effort to place the data in each part of the city. The index shows an indicator with 
five levels represented by a color scale from the turquoise to red ("Good" "Fair" "Poor" "Bad" and "Unhealthy"). 
The color-coding is consistent with the colors which are used in official sources, thereby avoiding onfusion any 
user who might consult official sources.

\begin{figure}[ht]
    \centering
    \includegraphics[width=12cm]{EAQI}
    \caption{EAQI Levels}
\end{figure}

These icons below are used to help the user have an immediate idea of the situation, since they are even more
decriptive and self-explanatory than just colors alone. Danger is, of course, almost universally indicated by 
the colour red, and is therefore used appropriately. However, no all cultures has the same perception about
the blue or green, missing what they indicate\\

\begin{figure}[ht]
    \centering
    \includegraphics[width=10cm]{EAQI_Icons}
    \caption{Iconografica Aire Guru}
\end{figure}

For the graphs that show variations in time,  we have chosen line graphs as being most appropriate,
since they show the continuous evolution over a period of time. To represent the different components of the AQI, we chose a
graph of stacked bars, since it is easy to see what proportion of the total AQI is formed by which pollutant. \\

\begin{figure}[ht]
    \centering
    \subfigure[AQI Evolution]
        {\includegraphics[width=5.75cm]{lineChart}}
        \hfill
    \subfigure [AQI components]
        { \includegraphics[width=5.25cm]{stakedBarChart}}
    \caption{Charts}
\end{figure}

In addition, to explain the concept of AQI and create an awarnes about the influence that air pollution has on us, Aire Guru provides a
glossary. This aids understanding of why air pollution should matter to us, with descriptions of the pollutants, medical complications, sources of contamination, the iconography used and
an explanation of what AQI is and how it is calculated. \\

\elsparagraph{Evaluation}  
\begin{itemize}
    \done The language used in the whole tool is a common language, avoiding the use of difficult scientific terminology but providing clearly described information to understand the situation.
    \done The most appropriate graphs for each type of data have been researched and chosen.
    \crossed Some specific terms could not be substituted as "Air Quality Index".
    \done The tools necessary to understand the concept have been provided. The European standard of air quality has been used to
         represent the values and offer the user resources on the page for their comprehension in addition to external resources.
\end{itemize}
\subsection{Let the world know that you exist}

No matter how optimized the representation of the data is and how available we make it, if users are not aware
of the existence of the resource, they can not use it.

\subsubsection*{Suggested strategies} 

The best way is to advertise the product in the right media with right format.

\subsubsection*{In the context of Aire Guru \ldots}

Our webtool is implemented for the city of Málaga, so we are currently working to publicize it in this city.
It is currently available in the open data portal in the web site tab (https://datosabiertos.malaga.eu/aplicaciones). \\

Aire Guru participated in the first open data reuse contest organized by the Málaga City Council (http://cemi.malaga.eu/es/novedades/detalle/1-Concurso-de-Reutilizacion-de-Datos-Abiertos-del-Ayuntamiento-de-Malaga)
and was a finalist in the web page category.

\begin{figure}[ht]
    \centering
    \subfigure[Advertising in the open data portal of Málaga]
        {\centering \includegraphics[width=6cm]{aireGuruFinalist}}
    \hfill
    \subfigure[Finalist]
        {\centering \includegraphics[width=5cm]{aireGuruFinalistCertificate}}
    \caption{I Contest of reuse of open data. Málaga's town hall}
\end{figure}