\subsection{Optimizacion de la accesibilidad}

Para ofrecer los datos a los usuarios de una manera directa, es recomendable utilizr una plataforma a la que el usuario
este familiarizado. Por ejemplo, es mas factible que el usuario se sienta mas atraido a acceder a los datos si no requiere
ninguna instalacion en sus dispositivos.\\


\noindent\textbf{\textit{Aire Guru} }\\


La herramienta Aire guru presenta la informacion en el idioma nativo de la ciudad y se utiliza un lenguaje sencillo y directo.
Ademas se utiliza el mismo estilo, colores e iconografia en todo el diseno para que el usuario se familierize rapidamente y pueda
prestar atencion al significado de los datos en vez de perderse en el diseno e intentar encontrar su significado.

Para garantizar el alcance a la mayoria de la poblacion, Aire Guru esta disponible en las direcciones webs https://www.aire.guru y https://www.airquality.guru.
Usa SSL que garantiza la encriptacion de los datos atraves de la red, ademas, cada vez mas navegadores intentan proteger a los usuarios y
solo muestran paginas que utilizar un metodo seguro. 

Aire Guru se comprende de distintas secciones disponibles para todos los usuarios. En la zona superior, podemos ver el mapa que mediante la seleccion de un punto, nos
mostrara informacion detallada de este punto. A continuacion, cuenta con una seccion que filtrara la informacion acorde a las preferencias del usuario
y para concluir, muestra el historial de los contaminantes desde 2018.
Ademas, para usuarios identificados, mostrara directamente la informacion de su localizacion y mostrara al usuario su historial personal
con la polucion a la que ha estado rodeado.

Como vemos, todos los usuarios pueden ver la informacion basica sin necesidad de aportar ningun dato o identificarse, sin la necesidad de 
realizar ninguna descarga o instalacion y hoy en dia, casi todo el mundo esta familiarizado con la navegacion web.

El workflow de las distintas secciones de Aire Guru esta detalladamente estudiada. Como vemos en la Figura X. Aire Guru. Landing page. Top section, el punto de partida es la localizacion que nos interesa,
a continuacion se muestra la informacion general de este punto, el AQI general, despues el AQI de cada uno de los contaminantes que componen el 
AQI general y por ultimo los valores numericos de cada uno de los contaminantes. Como vemos vamos de menos a mas detalle.

Por ultimo, Aire Guru cuenta con un glosario, donde explica que significa el indice de calidad del aire y como se calcula. 
Ademas, cuenta con multiples enlaces externos donde el usuario puede obtener mas informacion.
\begin{itemize}
    \item \textit{Evaluation}
\end{itemize}
Ofrecerles a los usuarios una pagina web libre de cargos, facilita que el usuario acceda directamente a la informacion
sin tener que realizar ninguna descarga. Al eliminar esta barrera, el usuario nos da un punto de confianza.
--reacio a identificarse
--y actualmente se esta trabajando en su traduccion al ingles, para cubrir un rango mas amplio de su poblacion, ya que esta ciudad es cada vez mas cosmopolita.
